\input{preambuloSimple.tex}
\graphicspath{ {./images/} }
\usepackage{subcaption}
\usepackage{hyperref}
\usepackage{soul}


%----------------------------------------------------------------------------------------
%	TÍTULO Y DATOS DEL ALUMNO
%----------------------------------------------------------------------------------------

\title{	
\normalfont \normalsize 
\textsc{\textbf{Introducción a la Ciencia de Datos (2019)} \\ Máster Oficial Universitario en Ciencia de Datos e Ingeniería de Computadores \\ Universidad de Granada} \\ [25pt] % Your university, school and/or department name(s)
\horrule{0.5pt} \\[0.4cm] % Thin top horizontal rule
\huge Trabajo Integrador: EDA, Clasificación y Regresión \\ % The assignment title
\horrule{2pt} \\[0.5cm] % Thick bottom horizontal rule
}

\author{Luis Balderas Ruiz \\ \texttt{luisbalderas@correo.ugr.es}} 
 % Nombre y apellidos 


\date{\normalsize\today} % Incluye la fecha actual

%----------------------------------------------------------------------------------------
% DOCUMENTO
%----------------------------------------------------------------------------------------

\begin{document}

\maketitle % Muestra el Título

\newpage %inserta un salto de página

\tableofcontents % para generar el índice de contenidos

\listoffigures

\listoftables

\newpage

%
%\begin{figure}[H] %con el [H] le obligamos a situar aquí la figura
%	\centering
%	\includegraphics[scale=0.6]{e1.png}  %el parámetro scale permite agrandar o achicar la imagen. En el nombre de archivo puede especificar directorios
%	\caption{Progresión de la imagen de E en cada iteración} 
%	\label{fig:e1}
%\end{figure}


%----------------------------------------------------------------------------------------
%	Introducción
%----------------------------------------------------------------------------------------

\section{Introducción}


El presente documento contiene los resultados obtenidos en el Trabajo Teórico/Práctico Integrador para la evaluación de la asignatura Introducción a la Ciencia de Datos. El trabajo está formado por tres apartados, a saber, análisis de datos (en adelante EDA), regresión y clasificación, centrado en dos conjuntos de datos: Wankara para regresión y Vowel para clasificación. Ambos dos forman parte del repositorio de Keel (\cite{keel}), el primero en \cite{wankara} y el segundo en \cite{vowel}. A continuación, describo la estructura del documento. En la primera sección, se desarrolla el análisis exploratorio de ambos datasets. A continuación, tras sacar las conclusiones correspondientes, ataco el problema de regresión y, por último, el de clasificación.






%----------------------------------------------------------------------------------------
%	Análisis de datos
%----------------------------------------------------------------------------------------

\section{Análisis exploratorio de datos (EDA)}

\subsection{Conjunto de datos Wankara}

\subsection{Conjunto de datos Vowel}

El presente conjunto de datos contiene información sobre el reconocimiento de las once vocales existentes en inglés por parte de 15 interlocutores independientes. Es un problema de clasificación de once clases (las once vocales inglesas) con trece características, diez de ellas reales y tres enteras. A pesar de ser numéricas, esas tres variables son realmente categóricas, dado que

\begin{itemize}
	\item TT (0/1): Indica si la instancia es de entrenamiento (0) o test (1).
	\item Sex (0/1): Indica el género del hablante en dicha instancia.
	\item SpeakerNumber [0,14]: Indica el interlocutor de la instancia.
\end{itemize}

Por tanto, esas variables se pueden interpretar como factores más que numéricas. De hecho, TT será ignorada en el EDA porque es conveniente realizar el estudio sobre los datos al completo. Por tanto, tras estas modificaciones contamos con diez variables reales (F0-F9), dos factores (sexo y número de interlocutor) y once clases (0-10), con un total de 990 instancias. 

Para explorar los datos, en primer lugar, calculo un resumen estadístico de cada variable y su dispersión. En las variables categóricas encontramos:

\begin{itemize}
	\item Cada interlocutor tiene 66 apariciones en el conjunto de datos.
	\item 528 de ellos son hombres y 462 mujeres.
\end{itemize}

Para las variables numéricas, los resultados son los siguientes:

\begin{table}[H]
	\resizebox{\textwidth}{!}{\begin{tabular}{|c|c|c|c|c|c|c|c|c|c|c|}
		\hline
		& \textbf{F0} & \textbf{F1} & \textbf{F2} & \textbf{F3} & \textbf{F4} & \textbf{F5} & \textbf{F6} & \textbf{F7} & \textbf{F8} & \textbf{F9} \\ \hline
		\textbf{Min}     & -5.211      & -1.274      & -2.487      & -1.409      & -2.127      & -0.836      & -1.537      & -1.293      & -1.631      & -1.68       \\ \hline
		\textbf{1st Qua} & -3.888      & 1.052       & -0.97575    & -0.0655     & -0.769      & 0.196       & -0.307      & -0.09575    & -0.704      & -0.548      \\ \hline
		\textbf{Mediana} & -3.146      & 1877        & -0.5725     & 0.4335      & -0.299      & 0.552       & 0.022       & 0.328       & -0.3025     & -0.1565     \\ \hline
		\textbf{Media}   & -3.204      & 1.882       & -0.50777    & 0.5155      & -0.3057     & 0.6302      & -0.004365   & 0.33655     & -0.30298    & -0.07134    \\ \hline
		\textbf{3th Qua} & -2.603      & 2.738       & -0.06875    & 1.096       & 0.1695      & 1.0285      & 0.2965      & 0.77        & 0.09375     & 0.371       \\ \hline
		\textbf{Max}     & -0.941      & 5.074       & 1.431       & 2.377       & 1.831       & 2.327       & 1.403       & 2.039       & 1.309       & 1.396       \\ \hline
		\textbf{SD}      & 0.8689872   & 1.1752720   & 0.7119483   & 0.7592613   & 0.6646023   & 0.6038711   & 0.4619268   & 0.5733020   & 0.5701616   & 0.6039855   \\ \hline
	\end{tabular}}
	\caption{Resumen estadístico y desviación típica de las características reales}
\end{table}
%----------------------------------------------------------------------------------------
%	Regresión
%----------------------------------------------------------------------------------------

\section{Problema de regresión: Wankara}





%----------------------------------------------------------------------------------------
%	Clasificación
%----------------------------------------------------------------------------------------

\section{Problema de clasificación: Vowel}


\newpage
\section{Bibliografía}

%------------------------------------------------

\bibliography{citas} %archivo citas.bib que contiene las entradas 
\bibliographystyle{plain} % hay varias formas de citar

\end{document}
