\input{preambuloSimple.tex}
\graphicspath{ {./images/} }
\usepackage{subcaption}
\usepackage{hyperref}
\usepackage{soul}


%----------------------------------------------------------------------------------------
%	TÍTULO Y DATOS DEL ALUMNO
%----------------------------------------------------------------------------------------

\title{	
\normalfont \normalsize 
\textsc{\textbf{Introducción a la Ciencia de Datos (2019)} \\ Máster Oficial Universitario en Ciencia de Datos e Ingeniería de Computadores \\ Universidad de Granada} \\ [25pt] % Your university, school and/or department name(s)
\horrule{0.5pt} \\[0.4cm] % Thin top horizontal rule
\huge Trabajo Integrador: EDA, Clasificación y Regresión \\ % The assignment title
\horrule{2pt} \\[0.5cm] % Thick bottom horizontal rule
}

\author{Luis Balderas Ruiz \\ \texttt{luisbalderas@correo.ugr.es}} 
 % Nombre y apellidos 


\date{\normalsize\today} % Incluye la fecha actual

%----------------------------------------------------------------------------------------
% DOCUMENTO
%----------------------------------------------------------------------------------------

\begin{document}

\maketitle % Muestra el Título

\newpage %inserta un salto de página

\tableofcontents % para generar el índice de contenidos

\listoffigures

\listoftables

\newpage

%
%\begin{figure}[H] %con el [H] le obligamos a situar aquí la figura
%	\centering
%	\includegraphics[scale=0.6]{e1.png}  %el parámetro scale permite agrandar o achicar la imagen. En el nombre de archivo puede especificar directorios
%	\caption{Progresión de la imagen de E en cada iteración} 
%	\label{fig:e1}
%\end{figure}


%----------------------------------------------------------------------------------------
%	Introducción
%----------------------------------------------------------------------------------------

\section{Introducción}


El presente documento contiene los resultados obtenidos en el Trabajo Teórico/Práctico Integrador para la evaluación de la asignatura Introducción a la Ciencia de Datos. El trabajo está formado por tres apartados, a saber, análisis de datos (en adelante EDA), regresión y clasificación, centrado en dos conjuntos de datos: Wankara para regresión y Vowel para clasificación. Ambos dos forman parte del repositorio de Keel (\cite{keel}), el primero en \cite{wankara} y el segundo en \cite{vowel}. A continuación, describo la estructura del documento. En la primera sección, se desarrolla el análisis exploratorio de ambos datasets. A continuación, tras sacar las conclusiones correspondientes, ataco el problema de regresión y, por último, el de clasificación.






%----------------------------------------------------------------------------------------
%	Análisis de datos
%----------------------------------------------------------------------------------------

\section{Análisis exploratorio de datos (EDA)}

\subsection{Conjunto de datos Wankara}

\subsection{Conjunto de datos Vowel}

El presente conjunto de datos contiene información sobre el reconocimiento de las once vocales existentes en inglés por parte de 15 interlocutores independientes. Es un problema de clasificación de once clases (las once vocales inglesas) con trece características, diez de ellas reales y tres enteras. A pesar de ser numéricas, esas tres variables son realmente categóricas, dado que

\begin{itemize}
	\item TT (0/1): Indica si la instancia es de entrenamiento (0) o test (1).
	\item Sex (0/1): Indica el género del hablante en dicha instancia.
	\item SpeakerNumber [0,14]: Indica el interlocutor de la instancia.
\end{itemize}

\begin{figure}[H] %con el [H] le obligamos a situar aquí la figura
	\centering
	\includegraphics[scale=0.7]{dist-sexos.png}  %el parámetro scale permite agrandar o chicar la imagen. En el nombre de archivo puede especificar directorios
	\caption{Distribución por sexos de los interlocutores} 
	\label{fig:dist-sex}
\end{figure}

Por tanto, esas variables se pueden interpretar como factores más que numéricas. De hecho, TT será ignorada en el EDA porque es conveniente realizar el estudio sobre los datos al completo. Por tanto, tras estas modificaciones contamos con diez variables reales (F0-F9), dos factores (sexo y número de interlocutor) y once clases (0-10), con un total de 990 instancias. 

Para explorar los datos, en primer lugar, calculo un resumen estadístico de cada variable y su dispersión. En las variables categóricas encontramos:

\begin{itemize}
	\item Cada interlocutor tiene 66 apariciones en el conjunto de datos.
	\item 528 de ellos son hombres y 462 mujeres.
\end{itemize}

Para las variables numéricas, los resultados son los siguientes:

\begin{table}[H]
	\resizebox{\textwidth}{!}{\begin{tabular}{|c|c|c|c|c|c|c|c|c|c|c|}
		\hline
		& \textbf{F0} & \textbf{F1} & \textbf{F2} & \textbf{F3} & \textbf{F4} & \textbf{F5} & \textbf{F6} & \textbf{F7} & \textbf{F8} & \textbf{F9} \\ \hline
		\textbf{Min}     & -5.211      & -1.274      & -2.487      & -1.409      & -2.127      & -0.836      & -1.537      & -1.293      & -1.631      & -1.68       \\ \hline
		\textbf{1st Qua} & -3.888      & 1.052       & -0.97575    & -0.0655     & -0.769      & 0.196       & -0.307      & -0.09575    & -0.704      & -0.548      \\ \hline
		\textbf{Mediana} & -3.146      & 1877        & -0.5725     & 0.4335      & -0.299      & 0.552       & 0.022       & 0.328       & -0.3025     & -0.1565     \\ \hline
		\textbf{Media}   & -3.204      & 1.882       & -0.50777    & 0.5155      & -0.3057     & 0.6302      & -0.004365   & 0.33655     & -0.30298    & -0.07134    \\ \hline
		\textbf{3th Qua} & -2.603      & 2.738       & -0.06875    & 1.096       & 0.1695      & 1.0285      & 0.2965      & 0.77        & 0.09375     & 0.371       \\ \hline
		\textbf{Max}     & -0.941      & 5.074       & 1.431       & 2.377       & 1.831       & 2.327       & 1.403       & 2.039       & 1.309       & 1.396       \\ \hline
		\textbf{SD}      & 0.8689872   & 1.1752720   & 0.7119483   & 0.7592613   & 0.6646023   & 0.6038711   & 0.4619268   & 0.5733020   & 0.5701616   & 0.6039855   \\ \hline
	\end{tabular}}
	\caption{Resumen estadístico y desviación típica de las características reales}
\end{table}

Como se puede observar, el rango y el dominio de cada variable es distinto, lo que podría condicionar el rendimiento de los posteriores algoritmos que utilizaremos. Por tanto, será necesario un reescalado de las mismas para evitar esa discriminación positiva de unas variables respecto a otras sólo por ser "mayores". \\

A continuación, presento las 10 variables numéricas con más detenimiento. Una de las características principales de este conjunto de datos es que, si se estudia como un todo, el comportamiento de las variables a veces puede parecer errático. Sin embargo, no se debe soslayar el hecho de que tenemos una variable categórica, la del sexo, que nos hace prácticamente crear dos conjuntos de datos cuasi-independientes: hombres y mujeres, donde sí que encontramos correlaciones y claves para entender el funcionamiento de las características. Como digo, presento cada una de las variables, primero estudiándola en conjunto y luego separando por sexo. Para comprobar si las variables siguen una distribución normal, se han utilizado el test de Shapiro-Wilk (\cite{10.1093/biomet/52.3-4.591}) y la corrección de Lilliefors del test de Kolmogorov-Smirnov (\cite{10.1080/01621459.1967.10482916}). Además, para estudiar el comportamiento tanto por sexo como por interlocutor, represento vía boxplots cada variable.
\newpage

\subsubsection{F0}

Presentamos la variable F0. En primer lugar, reflejo un resumen estadístico de la misma así como su histograma diferenciando entre hombres y mujeres.
\begin{itemize}
	\item Media: -3.204
	\item Mediana: -3.146
	\item Desviación típica: 0.8689872
	\item Rango: [-5.211,-0.941]
	\item Primer y tercer cuartiles: (-3.888,-2.603)
	\item Asimetría: 0.0662973
	\item Curtosis: -0.4974651
\end{itemize}

Por el valor de la curtosis, la variable es platicúrtica y muy ligeramente asimétrica por la derecha.

\begin{figure}[H] %con el [H] le obligamos a situar aquí la figura
	\centering
	\includegraphics[scale=0.6]{dist-F0.png}  %el parámetro scale permite agrandar o chicar la imagen. En el nombre de archivo puede especificar directorios
	\caption{Histograma de la variable F0} 
	\label{fig:hist-F0}
\end{figure}

Pasamos a comprobar si F0 se distribuye según una normal. Para ello, establecemos los tests de hipótesis de Shapiro-Wilk y Lilliefors (Kolmogorov-Smirnov) con los siguientes resultados:

\begin{figure}[H]
	\centering
	\begin{subfigure}{.5\textwidth}
		\centering
		\includegraphics[width=.7\linewidth]{sw-F0.png}
		\caption{Shapiro-Wilk}
		\label{fig:sw-F0}
	\end{subfigure}%
	\begin{subfigure}{.5\textwidth}
		\centering
		\includegraphics[width=.7\linewidth]{l-F0.png}
		\caption{Lilliefors}
		\label{fig:l-F0}
	\end{subfigure}
	\caption{Tests de normalidad sobre F0}
	\label{fig:normF0}
\end{figure}

Como los p-valores son menores que 0.05, rechazamos la hipótesis nula (la variable sigue una distribución normal). \\


A continuación presento los boxplots:

\begin{figure}[H]
	\centering
	\begin{subfigure}{.5\textwidth}
		\centering
		\includegraphics[width=.9\linewidth]{bps0.png}
		\caption{Por sexo}
		\label{fig:bps0}
	\end{subfigure}%
	\begin{subfigure}{.5\textwidth}
		\centering
		\includegraphics[width=.9\linewidth]{bpsn0.png}
		\caption{Por interlocutor}
		\label{fig:bpsn0}
	\end{subfigure}
	\caption{Boxplot para F0 estudiando sexos e interlocutores}
	\label{fig:bf0}
\end{figure}

Como se puede ver, existe un outlier en el interlocutor 7. Además, observamos que los valores para los hombres tienden a ser mayores que para las mujeres. \\

Si ahora estudiamos los tests de normalidad por sexos, encontramos los siguientes resultados:

\begin{figure}[H]
	\centering
	\begin{subfigure}{.5\textwidth}
		\centering
		\includegraphics[width=.6\linewidth]{swh-F0.png}
		\caption{Shapiro-Wilk}
		\label{fig:swh-F0}
	\end{subfigure}%
	\begin{subfigure}{.5\textwidth}
		\centering
		\includegraphics[width=.6\linewidth]{lh-F0.png}
		\caption{Lilliefors}
		\label{fig:lh-F0}
	\end{subfigure}
	\caption{Tests de normalidad sobre F0 (hombres)}
	\label{fig:normhF0}
\end{figure}

En este caso, aunque también se rechaza la hipótesis de normalidad, el test de Lilliefors arroja un resultado menor pero cercano a 0.05, por lo que parece acercarse más la variable en los hombres a una distribución normal que con el conjunto completo. \\

Para las mujeres,

\begin{figure}[H]
	\centering
	\begin{subfigure}{.5\textwidth}
		\centering
		\includegraphics[width=.6\linewidth]{swm-F0.png}
		\caption{Shapiro-Wilk}
		\label{fig:swm-F0}
	\end{subfigure}%
	\begin{subfigure}{.5\textwidth}
		\centering
		\includegraphics[width=.6\linewidth]{lm-F0.png}
		\caption{Lilliefors}
		\label{fig:lm-F0}
	\end{subfigure}
	\caption{Tests de normalidad sobre F0 (mujeres)}
	\label{fig:normmF0}
\end{figure}

se encuentran unos p-valores menores que para los hombres, por lo que también se rechaza la hipótesis de normalidad.

\subsubsection{F1}

Como en el apartado anterior, primero reflejo un resumen estadístico de la variable:

\begin{itemize}
	\item Media: 1.882
	\item Mediana: 1.877
	\item Desviación típica: 1.175272
	\item Rango: [-1.274,5.074]
	\item Primer y tercer cuartil: (1.052,2.738)
	\item Asimetría: -0.04269788
	\item Curtosis: -0.39925
\end{itemize}

Por el valor de la curtosis, la variable es platicúrtica y muy ligeramente asimétrica por la izquierda.

\begin{figure}[H] %con el [H] le obligamos a situar aquí la figura
	\centering
	\includegraphics[scale=0.6]{dist-F1.png}  %el parámetro scale permite agrandar o chicar la imagen. En el nombre de archivo puede especificar directorios
	\caption{Histograma de la variable F1} 
	\label{fig:hist-F1}
\end{figure}

Pasamos a estudiar la normalidad de la variable. Estos son los resultados de los tests:

\begin{figure}[H]
	\centering
	\begin{subfigure}{.5\textwidth}
		\centering
		\includegraphics[width=.7\linewidth]{sw-F1.png}
		\caption{Shapiro-Wilk}
		\label{fig:sw-F1}
	\end{subfigure}%
	\begin{subfigure}{.5\textwidth}
		\centering
		\includegraphics[width=.75\linewidth]{l-F1.png}
		\caption{Lilliefors}
		\label{fig:l-F1}
	\end{subfigure}
	\caption{Tests de normalidad sobre F1}
	\label{fig:normF1}
\end{figure}

Encontramos una discrepancia en los tests. Según el p-valor de Shapiro-Wilk, debemos rechazar la hipótesis de normalidad. Sin embargo, Lilliefors nos indica (0.2628) que no podemos rechazarla. Para esclarecer un poco más el comportamiento de F1, realizo la gráfica de densidad con una normal superpuesta con la media y desviación típica de F1:

 \begin{figure}[H] %con el [H] le obligamos a situar aquí la figura
 	\centering
 	\includegraphics[scale=0.6]{density-F1.png}  %el parámetro scale permite agrandar o chicar la imagen. En el nombre de archivo puede especificar directorios
 	\caption{Histograma con la densidad de F1 y normal para comparar} 
 	\label{fig:dense-F1}
 \end{figure}

así como un QQPlot (\cite{10.1093/biomet/55.1.1}):

\begin{figure}[H] %con el [H] le obligamos a situar aquí la figura
	\centering
	\includegraphics[scale=0.6]{qq-F1.png}  %el parámetro scale permite agrandar o chicar la imagen. En el nombre de archivo puede especificar directorios
	\caption{QQPlot de la variable F1} 
	\label{fig:qq-F1}
\end{figure}

A la luz de los resultados, podemos interpretar que F1 tiene un comportamiento muy parecido a la distribución normal. A continuación, presento los boxplots:

\begin{figure}[H]
	\centering
	\begin{subfigure}{.5\textwidth}
		\centering
		\includegraphics[width=.9\linewidth]{bps1.png}
		\caption{Por sexo}
		\label{fig:bps1}
	\end{subfigure}%
	\begin{subfigure}{.5\textwidth}
		\centering
		\includegraphics[width=.9\linewidth]{bpsn1.png}
		\caption{Por interlocutor}
		\label{fig:bpsn1}
	\end{subfigure}
	\caption{Boxplot para F1 estudiando sexos e interlocutores}
	\label{fig:bf1}
\end{figure}

Podemos apreciar outliers en los interlocutores 3,8,13 y 14. \\

Si ahora estudiamos los tests de normalidad por sexos, encontramos los siguientes resultados. Para los hombres

\begin{figure}[H]
	\centering
	\begin{subfigure}{.5\textwidth}
		\centering
		\includegraphics[width=.6\linewidth]{swh-F1.png}
		\caption{Shapiro-Wilk}
		\label{fig:swh-F1}
	\end{subfigure}%
	\begin{subfigure}{.5\textwidth}
		\centering
		\includegraphics[width=.6\linewidth]{lh-F1.png}
		\caption{Lilliefors}
		\label{fig:lh-F1}
	\end{subfigure}
	\caption{Tests de normalidad sobre F1 (hombres)}
	\label{fig:normhF1}
\end{figure}

Ambos tests nos llevan a rechazar la hipótesis de normalidad. En el caso de las mujeres

\begin{figure}[H]
	\centering
	\begin{subfigure}{.5\textwidth}
		\centering
		\includegraphics[width=.6\linewidth]{swm-F1.png}
		\caption{Shapiro-Wilk}
		\label{fig:swm-F1}
	\end{subfigure}%
	\begin{subfigure}{.5\textwidth}
		\centering
		\includegraphics[width=.6\linewidth]{lm-F1.png}
		\caption{Lilliefors}
		\label{fig:lm-F1}
	\end{subfigure}
	\caption{Tests de normalidad sobre F1 (mujeres)}
	\label{fig:normmF1}
\end{figure}

también rechazamos la hipótesis nula. Por tanto, en conjunto la variable se comporta según una normal pero por separado (hombres/mujeres) no.

\subsubsection{F2}

Presentamos la variable F2. En primer lugar, reflejo un resumen estadístico de la misma así como su histograma diferenciando entre hombres y mujeres.
\begin{itemize}
	\item Media: -0.50777
	\item Mediana: -0.5725
	\item Desviación típica: 0.7119483
	\item Rango: [-2.487,-1.431]
	\item Primer y tercer cuartiles: (-0.97575,-0.06875)
	\item Asimetría: 0.2352169
	\item Curtosis: -0.1575597
\end{itemize}

Por el valor de la curtosis, la variable es platicúrtica y ligeramente asimétrica por la derecha.

\begin{figure}[H] %con el [H] le obligamos a situar aquí la figura
	\centering
	\includegraphics[scale=0.6]{dist-F2.png}  %el parámetro scale permite agrandar o chicar la imagen. En el nombre de archivo puede especificar directorios
	\caption{Histograma de la variable F2} 
	\label{fig:hist-F2}
\end{figure}


Pasamos a estudiar la normalidad de la variable. Estos son los resultados de los tests:

\begin{figure}[H]
	\centering
	\begin{subfigure}{.5\textwidth}
		\centering
		\includegraphics[width=.7\linewidth]{sw-F2.png}
		\caption{Shapiro-Wilk}
		\label{fig:sw-F2}
	\end{subfigure}%
	\begin{subfigure}{.5\textwidth}
		\centering
		\includegraphics[width=.7\linewidth]{l-F2.png}
		\caption{Lilliefors}
		\label{fig:l-F2}
	\end{subfigure}
	\caption{Tests de normalidad sobre F2}
	\label{fig:normF2}
\end{figure}

Ambos tests nos indican que debemos rechazar la hipótesis de normalidad. En cuanto a los boxplots:

\begin{figure}[H]
	\centering
	\begin{subfigure}{.5\textwidth}
		\centering
		\includegraphics[width=.9\linewidth]{bps2.png}
		\caption{Por sexo}
		\label{fig:bps2}
	\end{subfigure}%
	\begin{subfigure}{.5\textwidth}
		\centering
		\includegraphics[width=.9\linewidth]{bpsn2.png}
		\caption{Por interlocutor}
		\label{fig:bpsn2}
	\end{subfigure}
	\caption{Boxplot para F2 estudiando sexos e interlocutores}
	\label{fig:bf2}
\end{figure}

Encontramos en F2 una gran concentración de outliers para los interlocutores 5,6,10,11 y 13. Es posible que las mediciones hayan sido erróneas. En cualquier caso, podría ser importante la aplicación de técnicas para la disminución de estos valores extraños. \\

Si ahora estudiamos los tests de normalidad por sexos, encontramos los siguientes resultados:
\begin{figure}[H]
	\centering
	\begin{subfigure}{.5\textwidth}
		\centering
		\includegraphics[width=.6\linewidth]{swh-F2.png}
		\caption{Shapiro-Wilk}
		\label{fig:swh-F2}
	\end{subfigure}%
	\begin{subfigure}{.5\textwidth}
		\centering
		\includegraphics[width=.6\linewidth]{lh-F2.png}
		\caption{Lilliefors}
		\label{fig:lh-F2}
	\end{subfigure}
	\caption{Tests de normalidad sobre F2 (hombres)}
	\label{fig:normhF2}
\end{figure}






\begin{figure}[H]
	\centering
	\begin{subfigure}{.5\textwidth}
		\centering
		\includegraphics[width=.6\linewidth]{swm-F2.png}
		\caption{Shapiro-Wilk}
		\label{fig:swm-F2}
	\end{subfigure}%
	\begin{subfigure}{.5\textwidth}
		\centering
		\includegraphics[width=.6\linewidth]{lm-F2.png}
		\caption{Lilliefors}
		\label{fig:lm-F2}
	\end{subfigure}
	\caption{Tests de normalidad sobre F2 (mujeres)}
	\label{fig:normmF2}
\end{figure}

Para ambos subconjuntos, los tests nos indican que debemos rechazar la hipótesis de normalidad.


%----------------------------------------------------------------------------------------
%	Regresión
%----------------------------------------------------------------------------------------

\section{Problema de regresión: Wankara}





%----------------------------------------------------------------------------------------
%	Clasificación
%----------------------------------------------------------------------------------------

\section{Problema de clasificación: Vowel}


\newpage
\section{Bibliografía}

%------------------------------------------------

\bibliography{citas} %archivo citas.bib que contiene las entradas 
\bibliographystyle{plain} % hay varias formas de citar

\end{document}
